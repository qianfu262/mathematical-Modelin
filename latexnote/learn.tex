\documentclass[12pt, a4paper, oneside]{ctexart}
\usepackage{threeparttable}%插入表格的
\usepackage{threeparttable}%插入表格的脚注
%%%
\usepackage{float}%控制图片位置
\usepackage{graphicx} %插入图片
\usepackage{subfigure} %控制图片位置
%%%
\usepackage{amsmath}
\usepackage{amssymb}
%%%
\usepackage{listings} %插入代码
\usepackage{appendix} %插入附录

\begin{document}
\tableofcontents %自动生成目录
\newcommand{\upcite}[1]{\textsuperscript{\cite{#1}}} %引用的格式
\newpage %生成新的一页

\section{一级标题}
\hspace{2.65mm} %缩进
这是文章内容
\vspace{2mm} %段内间隔的控制

\subsection{二级标题}

\subsubsection{三级标题}




\begin{itemize}
    \item 无序列表1
    \item 无序列表2
    \item 无序列表3
\end{itemize}


\begin{enumerate}
    \item 有序列表1
    \item 有序列表2
    \item 有序列表3
\end{enumerate}


\textbf{加粗} \\
\textit{斜体} \\
\underline{下划线} \\


%插入单张图片

\begin{figure}[H]
\centering 
%\includegraphics[width=6cm]{1.jpg}
\caption{图片标题}
\end{figure}

% 多张图片
\begin{figure}[H]

\centering
%\subfigure[图片1]{\includegraphics[width=3cm]{1.jpg}}
%\subfigure[图片2]{\includegraphics[width=3cm]{1.jpg}}
%\subfigure[图片3]{\includegraphics[width=3cm]{1.jpg}}
\caption{图片标题}
\end{figure}


%带标号的公式
\begin{equation}
y=x^{2}+1
\end{equation}

%无标号的公式
$$y=x^{2}+1$$

%插入图表,标准的三线表

\begin{table}[H]%这个H进行一个位置的设定
\centering
\caption{符号说明}%表格的标题
\begin{tabular}{cc}%第一个C表示左边居中显示,
    \hline %插入横线
    \makebox[0.3\textwidth][c]{符号}  &  \makebox[0.4\textwidth][c]{意义} \\ \hline
        $L_n$  & 经度\\ 
        $L_a$  & 纬度\\ \hline
\end{tabular}
\begin{tablenotes}
    \footnotesize
    \item \textbf{注:由于篇幅有限,仅展示部分数据}
\end{tablenotes}
\end{table}


%参考文献的插入
这里引用参考文献\upcite{1}
\begin{thebibliography}{9} %宽度为9
    \bibitem{1} 姜启源,数学模型(第四版) [M] 北京 高等教育出版社 2011
    \bibitem{2} 参考文献2
\end{thebibliography}


\newpage
\begin{appendices}

\section{二分法}
\lstset{language=Python}
\begin{lstlisting}
def binary_search(arr, target):
    left, right = 0, len(arr) - 1
    while left <= right:
        mid = (left + right) // 2
        if arr[mid] == target:
            return mid
        elif arr[mid] < target:
            left = mid + 1
        else:
            right = mid - 1
    return -1
\end{lstlisting}

\end{appendices}


\end{document}